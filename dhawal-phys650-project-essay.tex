\documentclass{article}
\usepackage{amssymb,amsthm,amsmath,quantikz}
\usepackage{stmaryrd}

\begin{document}
\begin{abstract}
This is an essay submitted for step 2 of the project for
course PHYS/CISC650 - Introduction to 
Quantum Computing and Quantum Information Theory taught 
by Prof Alexei Kananenka at University of Delaware in Fall 2025.
I have explained a few quantum error correction algorithms in this paper. The 
content is sourced mainly from \cite{roffe}. Figures are as attributed
but were texified manually by me.
\end{abstract}


The current trends in quantum computing make use of photons \cite{photon-qubit}, 
trapped ions \cite{trapped-ion-hyperfine-qubit},
superconducting qubits and spins in semiconductors to act as 
qubits. A common shortcoming of all these technologies is scalability and
fidelity. 
\emph{Scalability} - Current quantum computing devices operate with on
the order of a hundred qubits with approximately 20–30
gate operations and are not capable of error correction \cite{quantum-advantage%
-threshold}.
\emph{Fidelity} - It is not possible to isolate these qubits from the environment easily.
This results in several errors acting on the qubits such as $X,$ $Y$ and $Z$
errors. The $X$ error is a bit flip error which also occurs in classical
information processing systems. Quantum information processors are
different in that they also suffer from phase errors like $Y$ and $Z.$
The third problem is that it is difficult to construct an information processor
with several hundreds of too many qubits which interact with 
each other coherently. The advantage of quantum computing is that
information is stored not simply in the bits (qubits) but largely in
the probability correlation representing the probability of each qubit
occurring in a certain state. This makes quantum systems susceptible to
decoherence. This means that if the entanglement breaks down, then the 
information is lost. 

To combat these issues, Peter~Shor proposed the first quantum error
correction scheme in 1995 \cite{shor-error-correction}. This method
demonstrated how quantum information can be redundantly stored
and error be suppressed. In this paper, I will describe some
modern quantum error correction codes such as the stabiliser
code.

\section{Example - The Three Qubit Error Correction Code}
In this section I first describe the two qubit error detection code
which utilises projective measurements on the syndrome qubit.
Then I describe the three qubit error correction code.

Classical error correction codes depend on the reliable duplication
of the source information into multiple instances, and using more resources
to represent the duplicated information. Quantum information suffers from
lack of such reliability 
and the impossibility of doing so described in \cite{no-cloning}, as the 
no-cloning theorem raises
challenges in error correction protocols. The smallest correction code
which can correct all errors on a single qubit
uses 5 qubits \cite{Chao_2018}.
Any state $ \psi$
can not be cloned into another qubit simultaneously.
the operation
$$ \psi \xrightarrow{\text{two-qubit encoder}} \psi \otimes \psi$$
cannot be possibly done using any Unitary operator. Since all quantum gates
are described by unitary transformations, this operation is not
possible to be done with any quantum information processor at all.

If such an operator $U$ existed, then
$$A \ket{0} = \ket{00} \qquad \text{and} \qquad
A \ket{1} = \ket{11}.$$
So, 
\begin{flalign*}
	A ( \alpha \ket{0} + \beta \ket{1}) =
	\alpha^2 \ket{00} + \alpha \beta (\ket{01} + \ket{10})
	+ \beta^2 \ket{11} \\
	\not 
	= \alpha \ket{00} + \beta \ket{11}
	= \alpha( A \ket{0}) + \beta (A \ket{1}).
\end{flalign*}
This shows that $A$ cannot even be a linear operator, let alone a 
unitary. This theorem and proof are sourced from \cite{no-cloning}.
Even though this proof is written as if for two states,
it forbids cloning for arbitrary states in any system.

The means to circumvent this error is to pass the state
through a two qubit encoder 
$$ \alpha \vert 0 \rangle  + \beta \vert 1 \rangle
\xrightarrow {\text{two-qubit encoder}} \alpha \vert 00 \rangle
+ \beta \vert 11 \rangle
= \alpha \vert 0 \rangle_{L} \beta \vert 1 \rangle_{L}.$$
An error $X_1,$ respectively $X_2,$ which flips the first, respectively
second bit, sends the state $ \psi$ into the subspace
$\mathcal{F}$ spanned by $ \vert 01\rangle$ and $\vert 10 \rangle.$ Two 
qubit encoding thus 
renders both errors indistinguishable. It is thus capable of 
detecting errors but not of correcting them.


The code subspace $\mathcal{C}$ is the span of $ \vert 00 \rangle$
and $\vert 11 \rangle.$ Since $\mathcal{C}$ and $\mathcal{F}$ are
mutually orthogonal subspaces it is possible to make a projective
measurement on one of the qubits without destroying any of the
information.

\section{The projective measurement $Z_1 Z_2.$}
The $Z_1Z_2$ gate is described as follows
$$
Z_1 Z_2 = 
\begin{bmatrix}
	1 & 0 & 0 & 0 \\
	0 & -1 & 0 & 0 \\
	0 & 0 & -1 & 0 \\
	0 & 0 & 0 & -1 
\end{bmatrix}.
.
$$
If an error occurs, then $ \psi \in \mathcal{F}$
$$
\vert 01 \rangle
= \begin{bmatrix}
	0 \\ 0 \\ 1 \\ 0
\end{bmatrix}.
$$

Thus $ Z_1 Z_2 \vert 10 \rangle$ is given by

$$
\begin{bmatrix}
	1 & 0 & 0 & 0 \\
	0 & -1 & 0 & 0 \\
	0 & 0 & -1 & 0 \\
	0 & 0 & 0 & -1 
\end{bmatrix}
\begin{bmatrix}
	0 \\ 0 \\ 1 \\ 0
\end{bmatrix}
= -
\begin{bmatrix}
	0 \\ 0 \\ 1 \\ 0
\end{bmatrix}
= - \psi.
$$

Similarly, if $$\psi = \ket {01 } = 
\begin{bmatrix}
	0  \\ 1 \\ 0 \\ 0 \\	
\end{bmatrix},
$$
then
$$
Z_1 Z_2 \psi = 
\begin{bmatrix}
	1 & 0 & 0 & 0 \\
	0 & -1 & 0 & 0 \\
	0 & 0 & -1 & 0 \\
	0 & 0 & 0 & -1 
\end{bmatrix}
\begin{bmatrix}
	0  \\ 1 \\ 0 \\ 0 \\	
\end{bmatrix}
=
\begin{bmatrix}
	0  \\ -1 \\ 0 \\ 0 \\	
\end{bmatrix}
= - \psi.$$
Thus errors are detected by presence of a state in the characteristic
subspace corresponding to $-1.$
The fact that there are only two characteristic spaces, of values 1 and -1, can
be restated as follows. All the results of a single error on one qubit
is anticommutative with one of the projective measurements.

\section{Three Qubit error correction code}
Akin to the two qubit code, this algorithm also utilises projective measurements
which act on the main state and extract two qubits in the syndrome.
Similar the the two qubit encoder, this algorithm prepares the state in the first
qubit $ \psi = \alpha \vert 0 \rangle + \beta \vert 1 \rangle $ by duplicating it into the second and third qubits.
The latter two are prepared in the state zero and encoded as:
$$
\operatorname{CNOT}_{1,2} \operatorname{CNOT}_{1,3} \psi \otimes \vert 0 \rangle \otimes \vert 0 \rangle
= \alpha \vert 000 \rangle + \beta \vert 111 \rangle.$$
Following the error, the state enters one of the four orthogonal subspaces
$\mathcal{C} = \operatorname{span}\{ \vert 000 \rangle, \vert 111 \rangle \},$
$\mathcal{F}_1 = \operatorname{span} \{ \vert 100 \rangle, \vert 011 \rangle \},$
$\mathcal{F}_2 = \operatorname{span} \{ \vert 010 \rangle, \vert 101 \rangle \},$
and $\mathcal{F}_3 = \operatorname{span} \{ \vert 001 \rangle, \vert 110 \rangle \}.$
Following the duplication encoding, $ \psi$ enters $\mathcal{C}$ and stays in
this subspace $\mathcal{C}.$ if no error occurs. In case $X_1,$ $X_2,$ respectively
$X_3$ error occurs, it enters $\mathcal{C}_1,$ $\mathcal{C}_2,$ respectively
$\mathcal{C}_3.$
$$Z_1 Z_2 = \begin{bmatrix}
	1 & 0 & 0 & 0 & 0 & 0 & 0 & 0 \\
	0 & 1 & 0 & 0 & 0 & 0 & 0 & 0 \\
	0 & 0 & -1 & 0 & 0 & 0 & 0 & 0 \\
	0 & 0 & 0 & -1 & 0 & 0 & 0 & 0 \\
	0 & 0 & 0 & 0 & -1 & 0 & 0 & 0 \\
	0 & 0 & 0 & 0 & 0 & -1 & 0 & 0 \\
	0 & 0 & 0 & 0 & 0 & 0 & 1 & 0 \\
	0 & 0 & 0 & 0 & 0 & 0 & 0 & 1 \\
\end{bmatrix}.$$
As this matrix leaves 
$$
\begin{bmatrix}
	1 \\ 0 \\ 0 \\ 0 \\ 0 \\ 0 \\ 0 \\ 0
\end{bmatrix},
\begin{bmatrix}
	0 \\ 1 \\ 0 \\ 0 \\ 0 \\ 0 \\ 0 \\ 0
\end{bmatrix},
\begin{bmatrix}
	0 \\ 0 \\ 0 \\ 0 \\ 0 \\ 0 \\ 1 \\ 0
\end{bmatrix},
\begin{bmatrix}
	0 \\ 0 \\ 0 \\ 0 \\ 0 \\ 0 \\ 0 \\ 1 
\end{bmatrix}
$$
unchanged, these are in the characteristic subspace of $Z_1 Z_2.$
The $-1$ subspace of $Z_1Z_2$ is spanned by
$$
\begin{bmatrix}
	0 \\ 0 \\ 1 \\ 0 \\ 0 \\ 0 \\ 0 \\ 0 
\end{bmatrix},
\begin{bmatrix}
	0 \\ 0 \\ 0 \\ 1 \\ 0 \\ 0 \\ 0 \\ 0 
\end{bmatrix},
\begin{bmatrix}
	0 \\ 0 \\ 0 \\ 0 \\ 1 \\ 0 \\ 0 \\ 0 
\end{bmatrix},
\begin{bmatrix}
	0 \\ 0 \\ 0 \\ 0 \\ 0 \\ 1 \\ 0 \\ 0 
\end{bmatrix}.$$
Thus the operation $C_{Z_1Z_2}$ sends the first syndrome qubit $A_1$
from state $ \vert + \rangle$ to $ \vert - \rangle$ if the first two qubits are
in the $-1$ characteristic space and leaves it unchanged if it is in the $+1$ 
characteristic state.

So $ H A_1$ measures $1$ if $ \psi$ is in $\mathcal{F}_1$ or in $\mathcal{F}_2,$
the ones where first two qubits are different. Similarly, $HA_2$ measures $1$
if the second two qubits are different. If both happen, then the state is in $\mathcal{F}_2.$
The other cases are 
$$
\begin{array}{ll}
	space & A_1 A_2 \\
	\mathcal{C} & 00 \\
	\mathcal{F}_1 & 10 \\
	\mathcal{F}_2 & 11 \\
	\mathcal{F}_3 & 01 \\
\end{array}.
$$
The three qubit correction code only protects against single bit flips.
It does not protect against multiple bit flips or $Z$ errors.

\section{Stabiliser Codes}
The three qubit code is an example of a stabiliser code. In general,
an $k$ qubit code can be entangled with $m=n-k$ redundancy qubits and 
stabiliser measurements can be performed on the first $k$ qubits.
Then the result of each of the projective measurements $P_i$ can be recorded
in the $A_i$ syndrome qubit. The group $\mathcal{G}_n$ of Pauli operators
of weight upto $n$ consists of elements such as $I \otimes X \otimes Z \otimes
X.$ This operator is an element of $\mathcal{G}_4$ and has weight 3 because 
it has 3 non identity measurements. An element of $\mathcal{G}_n$
is called a stabiliser if it leaves the codespace unchanged.
The stabilisers $\{ P_i \}$ are selected such that
they are all elements of the Pauli group $\mathcal{G}_n$ and leave the
codespace unchanged. Moreover, it is essential that the stabilisers all 
commute with each other. This ensures that the measurements
can be performed in any order or even simultaneously.
	$(n,k,d)$ notation
\begin{enumerate}
	\item $n$ is the number of bits per codeword.
	\item $k$ is the number of encoded bits.
	\item $d$ is the code distance
\end{enumerate}
Here $d$ is the number minimum size of an error which goes undetected.

\section{The $ \llbracket 4,2,2 \rrbracket $ detection code}
The $ \llbracket 4,2,2 \rrbracket $ detection code is the smallest error correction code 
which can detect $X$ type as well as $Z$ type errors. However, it cannot correct
these errors though.


\begin{figure}
\begin{center}
\begin{quantikz}
	\lstick{ $ \vert \psi \rangle_1 $ } & \ctrl{2} & &
					    & & \targ{} & 
	\midstick[4,brackets=none] { $ \vert \psi_1 \psi_2 \rangle_L $ } 
					    & \gate[4]{E} &  & 
	\gate[4]{Z} & \gate[4]{{ X }} & &  & \\
	\lstick{ $ \vert \psi \rangle_2 $ } & & \ctrl{1} & & \targ{} & & 
					    & & & & & & & \\
	\lstick{ $ \vert 0 \rangle_3 $ } & \targ{} & \targ{} & \targ{} 
					 & & & & & & & & & & \\
	\lstick{ $ \vert 0 \rangle_4 $ } & \gate{H} & & \ctrl{-1} & \ctrl{-2} &
	\ctrl{-3} & & 
		  &  & & & & & \\
	\setwiretype{n} & & & & & & 
	& \lstick{ $ \vert 0 \rangle _{A_1 } $  } 
	& \gate{H} \setwiretype{q}  
	& \ctrl{-4} & & \gate{H} & \meter{} \\
	\setwiretype{n} & & & & & & 
	& \lstick{ $ \vert 0 \rangle _{A_2 } $  } 
	& \gate{H} \setwiretype{q}  
	& & \ctrl{-4} & \gate{H} & \meter{} \\
\end{quantikz}
\caption{Circuit for the $\llbracket 4,2,2 \rrbracket$ algorithm. %
source: \cite[figure 5]{roffe}}
\end{center}
\end{figure}

The encoder sends $ \vert 00 \rangle$ to 
$ \frac{1}{\sqrt{2}} ( \vert 0 0 0 0 \rangle + \vert 1  11 1 \rangle) ,$
$\vert 01 \rangle$ to
\begin{align*}
	{}& \operatorname{CNOT}_{4,1} \operatorname{CNOT}_{4,2} \operatorname{CNOT} _{4,3} \vert 0 1 1 + \rangle \\
	={}& \operatorname{CNOT}_{4,1} \operatorname{CNOT}_{4,2} \frac{1}{\sqrt{2}} ( \vert 0 1 1 0 \rangle + 
	\vert 0 1 0 1 \rangle ) \\
	={}& \operatorname{CNOT}_{4,1} \frac{1}{\sqrt{2}} 
	( \vert 0 1 1 0 \rangle + \vert 0 0 0 1 \rangle ) \\
	={}& \frac{1}{\sqrt{2}} ( \vert 0 1 1 0 \rangle + \vert 1 0 0 1 \rangle ) \\
\end{align*}
Essentially, the first syndrome qubit stores the parity of the data qubits and the 
second one entangles the first three with their flips.
So the codespace of this algorithm is
$$ \mathcal{C} = \operatorname{span}
\begin{Bmatrix}
 \vert 0 0 \rangle_{L} = \frac{1}{\sqrt{2}} ( \vert 0 0 0 0 \rangle
 + \vert 1 1 1 1 \rangle ) \\
 \vert 0 1 \rangle_{L} = \frac{1}{\sqrt{2}} ( \vert 0 1 1 0 \rangle
 + \vert 0 1 1 0 \rangle ) \\
 \vert 1 0 \rangle_{L} = \frac{1}{\sqrt{2}} ( \vert 1 0 1 0 \rangle
 + \vert 0 1 0 1 \rangle ) \\
 \vert 1 1 \rangle_{L} = \frac{1}{\sqrt{2}} ( \vert 1 1 0 0 \rangle
 + \vert 1 1 0 1 \rangle ) \\
\end{Bmatrix}
.$$
This space is stabilised by $X_1 X_2 X_3 X_4$ and $Z_1 Z_2 Z_3 Z_4.$
Also, the result of an $X$ flip is to make the first syndrome qubit 1, of a
$Z$ error is to make the second syndrome qubit 1. Thus the extraction 
of a syndrome allows detection of some error but does not tell us which error
occurred. This is also consistent with the equation
$$d = 2t + 1,$$
as the highest weight of logical operator in this algorithm is 2,
$d = 2.$ So $t=0$, that means it can not correct errors.

The algorithm which can correct all errors one one logical qubit
is the $\llbracket 5,1,3 \rrbracket$ algorithm.
Some errors can propagate to correlated errors of higher weight
on the data. 
This algorithm works by recording such errors on a flag which is
switched from $\ket{+}$ to $\ket{-}$ if such a propagative error
has occured.

\section{The Shor $\llbracket 9,1,3 \rrbracket$ Code}
The code works on one logical qubit and encodes it into 9 qubits.
This is done by concatenating the codeword with itself three times.
So the codeword $ \ket {-}_{3p} = \frac{1}{2} (\ket{000} + \ket{111})$ is 
transformed to $$\ket{-}_{3p} \ket{-}_{3p} \ket{-}_{3p}
 = \frac{1}{\sqrt{8}} (( \ket{000} - \ket{111}) ( \ket{000} - \ket{111}) 
( \ket{000} - \ket{111}) ).$$
The code $\ket{+}_{3p}$ is transformed similarly, and
the codespace is
$$ \mathcal{C}_{ \llbracket 9, 1, 3 \rrbracket }
 = \operatorname{span}
 \left\{ 
	 \begin{array}{l}
	 \ket{0}_{9} = \frac{1}{\sqrt{8}}  
	 \left( ( \ket{000} + \ket{111}) ( \ket{000} + \ket{111})  
	 (\ket{000} + \ket{111} ) \right)  \\
	 \ket{1}_{9} = \frac{1}{\sqrt{8}}  
	 \left( ( \ket{000} - \ket{111}) ( \ket{000} - \ket{111}) 
	( \ket{000} - \ket{111}) \right)  
 \end{array}
 \right\} .$$
 The stabilisers of this space are
 $$\mathcal{S}_{ \llbracket 9,1,3 \rrbracket}
 =  
 \begin{array}{l}
 \langle Z_1 Z_2, Z_2 Z_3, Z_4Z_5, Z_5 Z_6, Z_7 Z_8, Z_8Z_9,\\
 X_1 X_2 X_3 X_4 X_5 X_6 , X_4 X_5 X_6 X_7 X_8 X_9 \rangle.
 \end{array}$$
 Note that these are simply stabilisers of the individual components
 which are being concatenated together.


If the $X_5$ occurs, then $Z_4Z_5$ and $Z_5Z_6$ send the state
$ \psi = (\ket{000} - \ket{111}(\ket{000} - \ket{111}))(\ket{000} 0 \ket{111})$
(for example)
to the $-1$ characteristic space. All the other stabilisers
send it to the $+1$ subspace. So the syndrome extracted is
$001100000.$ This is because 
\begin{align*}
X_5 \ket{0}_{3p}
={}& X_5 \left( 
\frac{1}{\sqrt{8}}  
	 \left( ( \ket{000} + \ket{111}) ( \ket{000} + \ket{111})  
	 (\ket{000} + \ket{111} ) \right)  
 \right) \\
	={}&\left( 
\frac{1}{\sqrt{8}}  
	 \left( ( \ket{000} + \ket{111}) ( \ket{010} + \ket{101})  
	 (\ket{000} + \ket{111} ) \right)  
\right) .
\end{align*}

The $Z_4Z_5$ operator acts on this state to make it
$$
\frac{1}{\sqrt{8}} \left( 
	(\ket{000} + \ket{111}) (-\ket{010} - \ket{101})
	(\ket{000} + \ket{111})
\right) .$$
upon entanglement with a syndrome qubit corresponding to $Z_4Z_5,$
$ \psi \otimes \ket{+}_{4} = \frac{1}{\sqrt{2}} (\psi \otimes \ket{0} + 
\psi \otimes \ket{1}) $ transforms to
$Z_4 Z_5 \psi \otimes \ket{+}_4
= \frac{1}{\sqrt{2}} (\psi \otimes \ket{0}
- \psi \otimes \ket{1})
= \psi \otimes \ket{-}_4.$ The syndrome $\ket{+}_4$ is passed through the
Hadamard gate and extracted as a 0. Note that the key here is
that $Z_4Z_5$ acts on it only if the syndrome qubit is in state 1.
In this way the other syndrome bits can be extracted for all the states in the 
codespace.
If the syndrome reads $001100000$ then the error recovery is done
by applying $Z_4Z_5$ and $Z_5Z_6$ on the state.
In this way it can correct alll single qubit errors
having distance $d=3.$

\section{Scalability}
Some of the challenges with Stabiliser codes are finding 
the sets of stabilisers, and finding codes which work well with
the chosen stabilisers.
Hamming Codes,
which are $\llbracket 2^r - 1, 2^r -1-2r, 3 \rrbracket$ algorithms
of the Stabliiser algorithms, can be scaled. g
Another modern class of scalable quantum error correction algorithms is
the Surface code algorithms. The Surface code algorithms belong to a class
of codes called the topological codes. Surface codes mitigate the problem
of finding stabilisers
problem by patching together smaller repeated
elements of smaller codes. Hence they can be scaled to larger codes
easily. Moreover, these only require interaction of qubits with near neighbours.
This is helpful because high fidelity interactions over long ranges are not easily
possible between qubits in most quantum computing platforms. This makes surface 
codes the current standard in quantum error correction. 
However, there are challenges to surface codes. It has a poor
encoding density. While the distance of the code can be increased, the cost
of doing so is a vanishing code rate. It is also highly resource intensive.


\section{Current and future trends}
The future of quantum computing requires processors which can satisfy
the following four requirements. 
The architecture should have a high noise threshold. It should enable implementation
of universal sets of logic gates. The processing 
should be possible with few physical qubits per logical qubits. 
The need for more qubits arises from the role of redundancy in error 
correction. The times required to execute the error correction should also
be low so as to be compatible with the main computation itself. 
These constraints are physical. 
Low density parity check codes are one such class of error correction codes
which satsify certain upper bounds on the number of qubits involved in 
each parity check and the number of parity checks as well. 
These factors bring a great deal of attention to these algorithms in the 
field of quantum error correction.
	
 




\bibliographystyle{plain}
\bibliography{myref.bib}

\end{document}

